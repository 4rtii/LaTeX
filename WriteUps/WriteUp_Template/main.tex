\documentclass[a4paper]{article} % Formato de plantilla que vamos a utilizar

\usepackage[utf8]{inputenc}
\usepackage[spanish]{babel}
\usepackage[margin=2cm, top=2cm, includefoot]{geometry} % Establecer márgenes
\usepackage{graphicx}                   % Para agregar imágenes
\usepackage[table,xcdraw]{xcolor}       % Usar colores
\usepackage[most]{tcolorbox}            % Agregar recuadros en el documento
\usepackage{fancyhdr}                   % Definir el estilo de la página
\usepackage[hidelinks]{hyperref}        % Gestión de hipervínculos
\usepackage{parskip}                    % Arreglar la tabulación del texto
\usepackage{smartdiagram}               % Agregar diagramas en el documento
\usepackage{listings}                   % Agregar código en el documento
\usepackage{zed-csp}                    % Agregar esquemas/tablas en el documento
% \usepackage[figurename=test]{caption}   % Para personalizar el nombre de la caption de las fotos


\title{WriteUp Template}
\author{Arturo Cantú (aka 4rtii)}


% Declaración de colores
\definecolor{greenCover}{HTML}{69A84F}


% Definición de variables
\newcommand{\platformLogo}{img/HTB-Logo.png}        % Hacer referencia a la plataforma que corresponda
\newcommand{\headerLogo}{img/HTB-Header.png}        % Hacer referencia a la plataforma que corresponda (imagen alargada)
\newcommand{\platformURL}{https://hackthebox.com}   % Colocar el enlace a la plataforma que corresponda

\newcommand{\machineName}{$<$machine-name$>$}              % Nombre de la máquina
\newcommand{\machineLogo}{img/preignitionLogo.png}  % Imagen de la máquina
\newcommand{\machineInfo}{img/preignitionInfo.png}  % Información relevante de la máquina víctima

\newcommand{\machineURL}{https://app.hackthebox.com/starting-point} % Colocar URL que corresponda
\newcommand{\machineIP}{$<$ip-address$>$}                % Dirección ip de la máquina víctima
\newcommand{\machineDiff}{$<$machine-difficulty$>$}                % Dificultad de la máquina víctima
\newcommand{\startDate}{date}           % Fecha


% Adicionales
\addto\captionsspanish{\renewcommand{\contentsname}{Índice}} % Cambio de nombre de 'Índice'
\setlength{\headheight}{40.2pt}         % Definir espaciado de la cabecera
\pagestyle{fancy}                       % Agregar la línea en la cabecera
\fancyhf{}                              % Limpiar cabecera
\lhead{\includegraphics[width=6cm]{\headerLogo}}        % Lado izquierdo de la cabecera
\rhead{\includegraphics[width=1.5cm]{\machineLogo}}     % Lado derecho de la cabecera
\renewcommand{\headrulewidth}{3pt}      % Anchura línea de cabecera
\renewcommand{\headrule}{\hbox to\headwidth{\color{greenCover}\leaders\hrule height \headrulewidth\hfill}} % Cambiar color de la línea

\renewcommand{\lstlistingname}{Código}  % Para cambiar la caption de los códigos


% Necesario para insertar código en el documento y que te detecte la sintaxis
\definecolor{codegreen}{rgb}{0,0.6,0}
\definecolor{codegray}{rgb}{0.5,0.5,0.5}
\definecolor{codepurple}{rgb}{0.58,0,0.82}
\definecolor{backcolour}{rgb}{0.95,0.95,0.92}

\lstdefinestyle{mystyle}{
    backgroundcolor=\color{backcolour},   
    commentstyle=\color{codegreen},
    keywordstyle=\color{magenta},
    numberstyle=\tiny\color{codegray},
    stringstyle=\color{codepurple},
    basicstyle=\ttfamily\footnotesize,
    breakatwhitespace=false,         
    breaklines=true,                 
    captionpos=b,                    
    keepspaces=true,                 
    numbers=left,                    
    numbersep=5pt,                  
    showspaces=false,                
    showstringspaces=false,
    showtabs=false,                  
    tabsize=2
}

\lstset{style=mystyle}

% Comienzo del documento
\begin{document}
    \fancyfoot[R]{\thepage}     % Colocar # de página a la derecha del footer
    \begin{titlepage}
        \centering
        \includegraphics[width=0.5\textwidth]{\platformLogo}\par\vspace{1cm} % Logo de la plataforma
        {\scshape\LARGE\textbf{Informe Técnico}\par}
        \vspace{0.3cm}
        {\Huge\bfseries\textcolor{greenCover}{Máquina \machineName}}
        \vfill\vfill
        \includegraphics[width=\textwidth,height=10cm,keepaspectratio]{\machineLogo}\par\vspace{1cm} % Logo de la máquina
        \vfill
        \begin{tcolorbox}[colback=red!5!white,colframe=red!75!black]    % Recuadro
            \centering
            \textbf{Dirección IP:} \machineIP\\\textbf{Dificultad:} \machineDiff\\\textbf{Creador:} $<$name$>$
        \end{tcolorbox}
        \vfill
        {\large \startDate\par}
        \vfill
    \end{titlepage}
    
    %------------------------------------- FIN PORTADA // INICIO ÍNDICE -------------------------------------
    \clearpage          % Salto de página
    \tableofcontents    % Índice (TOC)
    \clearpage          % Salto de página
    %--------------------------------------------- FIN ÍNDICE ---------------------------------------------
    
    \section{Antecedentes}
        El presente documento está escrito a modo de guía o referencia para todas aquellas personas que quieran hacer la máquina \textbf{\machineName} de la plataforma \href{\platformURL}{\textbf{Hack The Box}}. Cabe aclarar que la manera en la que yo resolví la máquina no es la definitiva, esto es sólo una referencia que les puede ser de ayuda en caso de que la necesiten.
        
        \vspace{0.2cm}
        
        \begin{figure}[h]
            \centering
            \includegraphics[width=\textwidth]{\machineInfo}     % Información relevante de la máquina víctima
            \caption{Máquina {\machineName}}
            \label{fig:preignition}
        \end{figure}
        
        \vspace{0.2cm}
        
        \begin{tcolorbox}[enhanced,attach boxed title to top center={yshift=-3mm,yshifttext=-1mm},  % Recuadro URL
            colback=blue!5!white,colframe=blue!75!black,colbacktitle=greenCover!80!black,
            title=Dirección URL,fonttitle=\bfseries,
            boxed title style={size=small,colframe=red!50!black} ]
            \centering
            \href{\machineURL}{\color{blue}{\machineURL}}
        \end{tcolorbox}
        
        \vspace{0.2cm}
     
    % ------------------------------------------ FIN ANTECEDENTES ------------------------------------------
        
    \section{Objetivos}
    Comprometer la máquina \textbf{\machineName} con el fin de llegar a ser el usuario con máximos privilegios utilizando técnicas de reconocimiento, escaneo, enumeración, análisis de vulnerabilidades, explotación y post-explotación.
    
    \vspace{0.3cm}
    
    \begin{figure}[h]
        \begin{center}
            \smartdiagramset{
                back arrow disabled=true, 
                text width=4cm, 
                module x sep=5cm,
                set color list={blue!40,yellow!40,red!40},
                uniform arrow color=true
                }
                \smartdiagram[flow diagram:horizontal]{
                Reconocimiento sobre el sistema,
                Análisis y detección de vulnerabilidades,
                Explotación de vulnerabilidades
                }
        \end{center}
        \caption{Flujo de trabajo}
        \label{dia:metodología}
    \end{figure}
    
    \clearpage
    
     % ------------------------------------------ FIN OBJETIVOS ------------------------------------------

    \section{Análisis de Vulnerabilidades}
        \subsection{Reconocimiento Inicial}
        \vspace{0.2cm}
        Se comenzó realizando una análisis inicial sobre el sistema, verificando que el sistema objetivo se encontrara accesible desde el segmento de red en el que se opera:
        
        \begin{lstlisting}[language=Bash, caption=Reconocimiento inicial sobre la máquina víctima]
user@bash:~$ ping -c 1 <ip-address>
insertar resultado del ping
        \end{lstlisting}
        
        \vspace{0.3cm}
        
        \subsection{Fase de escaneos}
        Una vez verificada la conectividad con la máquina víctima, se realizó un escaneo haciendo uso de la herramienta \textbf{nmap} para la detección de puertos abiertos, obteniendo los siguientes resultados:
        
        \vspace{0.2cm}
        
        \begin{lstlisting}[language=Bash, caption=Escaneo de puertos a la máquina víctima]
user@bash:~$ nmap -p- -sS --min-rate 5000 --open -n -vvv -Pn <ip-address> -oG allPorts
insertar captura nmap
        \end{lstlisting}
        
        \clearpage
        
    % -------------------------------- FIN ANÁLISIS DE VULNERABILIDADES --------------------------------
        
    \section{Pruebas}
    Para insertar código en el documento se tiene que hacer lo siguiente:
        
    \vspace{0.3cm}
        
    \begin{lstlisting}[language=Bash, caption=Script personalizado para ...]
#!/bin/bash
            
for port in $(seq 1 65535); do
    timeout 1 bash -c "echo '' > /dev/tcp/11.22.33.44/$port" 2>/dev/null && echo "Port $port - Open" &
done; wait
    \end{lstlisting}
        
    \vspace{0.3cm}
        
    Para agregar esquemas en el documento se tiene que hacer lo siguiente:
        
    \begin{schema}{TCP}
        Puertos
        \where
        21, 22, 25, 53, 80, 443
    \end{schema}
        
    \vspace{0.3cm}
        
    Para agregar una imagen que te ocupe todo el ancho de la hoja (o menos) se hace lo siguiente:
    \begin{figure}[h]
        \centering
        \makebox[\textwidth]{\includegraphics[width=\paperwidth]{img/preignitionInfo.png}}
        \caption{Imagen que te ocupa todo el ancho de la hoja (o menos)}
        \label{fig:test}
    \end{figure}
        
    \vspace{0.3cm}
        
    Hacer referencia a imágenes -$>$ figura \ref{fig:test} de la página \pageref{fig:test}
        
    \clearpage
    
     % ------------------------------------------ FIN { } ------------------------------------------
    
    \section{Créditos}
        \textbf{Autor:} Arturo Cantú (aka 4rtii)
        
        \textbf{Inspirado en:} Marcelo Vázquez (aka s4vitar): \href{https://youtube.com/watch?v=riNRHoEOBeU}{\color{blue}{Cómo crear un reporte profesional en LaTeX}}
        
        \textbf{Sitio Web:} \href{https://4rtii.github.io}{\color{blue}{4rtii.github.io}}
        
        \textbf{YouTube:} \href{https://youtube.com/c/4rtii}{\color{blue}{4rtii}}
        
        \textbf{Twitch:} \href{https://twitch.tv/4rtii_}{\color{blue}{4rtii_}}
        
        \textbf{GitHub:} \href{https://github.com/4rtii}{\color{blue}4rtii}
        
        \textbf{Hack The Box:} \href{https://app.hackthebox.com/profile/839583}{\color{blue}{4rtii}}
        
\end{document}
